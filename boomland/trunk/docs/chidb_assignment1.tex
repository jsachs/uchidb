\documentclass[10pt]{article}
\usepackage{fullpage}
\usepackage{graphicx}
\usepackage{url}

\newcommand{\chidb}{$\chi$\textsf{db}}

%opening
\title{\chidb{} Assignment I: B-Trees}

\date{}


\begin{document}
\pagestyle{empty}
\maketitle

In this first assignment, you will be implementing a C library called \textsf{libchidb} that provides \chidb{} B-Tree manipulation functions. Assignments II and III add more functionality to this library, such as a database machine and a query optimizer. This assignment is divided into the following steps:

\begin{enumerate}
\item Opening a \chidb{} file
\item Loading a B-Tree node from the file
\item Creating and writing a B-Tree node to disk
\item Manipulating B-Tree cells
\item Finding a value in a B-Tree
\item Insertion into a leaf without splitting
\item Insertion with splitting
\item Supporting index B-Trees
\end{enumerate}

All of these steps can be developed independently (the only exception is step 8, which is a cross-cutting feature). Nonetheless, testing a step will usually require that some of the previous steps be implemented (e.g., you can't find a value in a B-Tree if you can't open the file itself).


\section*{Before you get started}

Before you start working on this assignment, make sure read \emph{The \chidb{} File Format}. You are not expected to understand the entire document at first, but you should at least get a general feel for how a \chidb{} file is organized.


\section*{Getting the \chidb{} code}

To allow you to focus on the database-specific code, you are provided with code that handles some of the more tedious parts of manipulating a \chidb{} file (e.g., loading a page into memory, converting a \textsf{varint32} into a C integer, etc.). The code is available on the University of Chicago PhoenixForge SVN repository. If you are going to work on the code without using an SVN repository of your own, you can get the code by doing the following:

\begin{verbatim}
svn co https://phoenixforge.cs.uchicago.edu/svn/chidb/trunk chidb
\end{verbatim}

If you will be working on a copy of the \chidb{} code on an SVN repository of your own, we recommend that you use \texttt{svn import} and \texttt{svn export}. On a temporary directory, do the following:

\begin{verbatim}
svn export https://phoenixforge.cs.uchicago.edu/svn/chidb/trunk chidb
\end{verbatim}

This will create a directory called \texttt{chidb} with the \chidb{} code. Before looking at it, import it into your  repository like this:

\begin{verbatim}
svn import chidb https://repository.server.com/svn/myproject/chidb
\end{verbatim}

Note that this does not place your local \texttt{chidb} directory under version control. You will now have to check out your code into the local directory that you want to use as your working directory:

\begin{verbatim}
svn checkout https://repository.server.com/svn/myproject/chidb
\end{verbatim}

If you are unfamiliar with SVN, you can find a quick introduction to SVN at \url{http://svnbook.red-bean.com/en/1.5/svn.basic.in-action.html}. 

\section*{Overview of the \chidb{} code}

Once you have the \chidb{} code in a local directory, run \texttt{make} to build it (the code should compile correctly). This should result in a \texttt{libchidb.so} shared library file in the \chidb{} directory.

The code you are given is the following:

\begin{itemize}
\item[---] \texttt{src/libchidb/main.c} and \texttt{include/chidb.h}: the \chidb{} API, containing the only functions that external programs using \chidb{} should have access to.
\item[---] \texttt{include/chidbInt.h}: internal declarations and definitions.
\item[---] \texttt{src/libchidb/pager.[hc]}: the pager module.
\item[---] \texttt{src/libchidb/record.[hc]}: the database record module.
\item[---] \texttt{src/libchidb/util.[hc]}: miscellaneous helper functions.
\end{itemize}

Most of your work will be done on the files \texttt{src/libchidb/btree.[hc]}. The header file contains declarations of functions you will have to implement, and which are referred to in the rest of the document. The specification of what the functions must do is included in the source as comments.

The provided code also includes a set of automated tests (written with the cunit\footnote{\url{http://cunit.sourceforge.net}} library) in the \texttt{tests} directory. To build these tests, run \texttt{make tests}. These tests use the cunit\footnote{http://cunit.sourceforge.net} library, so you will need to have this library installed, including the development headers, to build and run the tests. To run the tests, run the \texttt{tests} executable in the \texttt{tests} directory. The provided tests only test the code that is provided to you. Additionally, the \chidb{} website provides a ``black box'' binary you can use to test your B-Tree implementation.

Note that to run the tests, or any executable that needs to dynamically link with \texttt{libchidb.so}, you need to set your \verb+LD_LIBRARY_PATH+ environment variable to the directory where the \texttt{libchidb.so} is located. For example:

\begin{verbatim}
export LD_LIBRARY_PATH=/home/jrandom/chidb
\end{verbatim}

\section*{Step 1: Opening a \chidb{} file}

Implement the following functions:

\begin{verbatim}
int chidb_Btree_open(const char *filename, chidb *db, BTree **bt)
int chidb_Btree_close(BTree *bt);
\end{verbatim}

Take into account that the \verb+chidb_Btree_open+ function can open an existing file, but can also create an empty database file if given the name of a file that does not exist. The latter functionality will not be possible until Step 3 is completed.

\section*{Step 2: Loading a B-Tree node from the file}

Implement the following functions:

\begin{verbatim}
int chidb_Btree_getNodeByPage(BTree *bt, npage_t npage, BTreeNode **node);
int chidb_Btree_freeMemNode(BTree *bt, BTreeNode *btn);
\end{verbatim}

\section*{Step 3: Creating and writing a B-Tree node to disk}

Implement the following function:

\begin{verbatim}
int chidb_Btree_newNode(BTree *bt, npage_t *npage, uint8_t type);
int chidb_Btree_initEmptyNode(BTree *bt, npage_t npage, uint8_t type);
int chidb_Btree_writeNode(BTree *bt, BTreeNode *node);
\end{verbatim}


\section*{Step 4: Manipulating B-Tree cells}

Implement the following functions:

\begin{verbatim}
int chidb_Btree_getCell(BTreeNode *btn, ncell_t ncell, BTreeCell *cell);
int chidb_Btree_insertCell(BTreeNode *btn, ncell_t ncell, BTreeCell *cell);
\end{verbatim}

Take into account that, once you've implemented \texttt{chidb\_Btree\_getCell} (which is the simpler of the two), you will be able to implement \texttt{chidb\_Btree\_find} or use the provided \texttt{chidb\_Btree\_print}. At that point, you will be able to open example database files (available at the \chidb{} website) and verify that you can correctly print out their contents or search for specific values.

\section*{Step 5: Finding a value in a B-Tree}

Implement the following function:

\begin{verbatim}
int chidb_Btree_find(BTree *bt, npage_t nroot, key_t key, uint8_t **data, uint16_t *size);
\end{verbatim}


\section*{Step 6: Insertion into a leaf without splitting}

Implement the following functions:

\begin{verbatim}
int chidb_Btree_insertInTable(BTree *bt, npage_t nroot, 
                              key_t key, uint8_t *data, uint16_t size);
int chidb_Btree_insert(BTree *bt, npage_t nroot, BTreeCell *btc);
int chidb_Btree_insertNonFull(BTree *bt, npage_t npage, BTreeCell *btc);
\end{verbatim}

Take into account that, at this point, \verb+chidb_Btree_insert+ will be little more than a call to \verb+chidb_Btree_insertNonFull+. Also, even if at this point you are only inserting in leaf nodes, that doesn't mean that your implementation shouldn't work on a database file that does have internal nodes. So, \verb+chidb_Btree_insertNonFull+ should still traverse the tree in search of the leaf node to insert the cell in (but assuming that splitting will not be necessary)

Since the \chidb{} file format is a subset of the SQLite format, once this step is completed you will be able to create a \chidb{} file and open it in SQLite, as long as you include a valid schema table in the file.

\section*{Step 7: Insertion with splitting}

Implement the following function:

\begin{verbatim}
int chidb_Btree_split(BTree *bt, npage_t npage_parent, npage_t npage_child, 
                                 ncell_t parent_cell, npage_t *npage_child2);
\end{verbatim}

Note that you will also have to modify \verb+chidb_Btree_insert+ and \verb+chidb_Btree_insertNonFull+ to split nodes when necessary.

\section*{Step 8: Supporting index B-Trees}

Supporting index B-Trees affects almost all of the previous functions. As such, this step can either be done from the very beginning (at the cost of complicating the implementation of the previous functions), or added at the end (simplifying the implementation of the previous functions, but adding more work at the end of this assignment).

Besides modifying the previous functions, you will also have to implement the following function:

\begin{verbatim}
int chidb_Btree_insertInIndex(BTree *bt, npage_t nroot, key_t keyIdx, key_t keyPk);
\end{verbatim}

\end{document}
